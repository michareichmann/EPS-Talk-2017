\begin{frame}{Motivation}

	\begin{itemize}
		\item innermost layers \ra highest radiation damage (\SIrange{100}{200}{\mega\hertz\per cm^2})
		\item current detector is designed to survive \SI{\sim12}{month} in High-Luminosity LHC
		\item \usebeamercolor[fg]{title} \textbf{\ra R\&D for more radiation hard detector designs and/or materials}
	\end{itemize}\vspace*{10pt}
	
	\uncover<2->{
		\textbf{\underline{Diamond as Detector Material:}}\vspace*{5pt}
		\begin{itemize}
			\itemfill
			\item properties 
			\begin{itemize}
				\itemfill
				\item radiation tolerance
				\item isolating material
				\item high charge carrier mobility
				\item \textcolor{RedOrange}{smaller signal than in silicon}
			\end{itemize}
		\end{itemize}}

% 	\uncover<3->{
% 		\begin{itemize}
% 			\item diamond pixel detectors in Pixel Luminosity Telescope (PLT) (installed 2010/2011)
% 			\begin{itemize}
% 				\item \textcolor{RedOrange}{\textbf{signal dependence on incident particle rate observed}} (single crystal)
% 			\end{itemize}
% 		\end{itemize}}
	
	\uncover<3->{
		\begin{itemize}
			\item investigation of the signal independence/dependence on incident particle flux in various detector designs:
			\begin{itemize}
				\itemfill
				\item pad \ra full diamond as single cell readout
				\item pixel \ra diamond sensor on pixel chips
				\item 3D \ra strip/pixel detector with clever design to reduce drift distance
			\end{itemize}

		\end{itemize}}
		
\end{frame}
